% \textbf{FIXME:  basically a placeholder; do not believe}

\svnidlong{$LastChangedBy$}{$LastChangedRevision$}{$LastChangedDate$}{$HeadURL: http://freevariable.com/dissertation/branches/diss-template/frontmatter/abstract.tex $}
\vcinfo{}

Despite tremendous progress in quantum information processing towards prototypical devices and early attempts at showing quantum utility, there remain a number of outstanding technical challenges. State of the art quantum computers based on neutral atoms take up substantial space, harkening back to the vacuum tube era of classical computing. Reducing the spatial requirements of these machines by transitioning to more integrated architectures is an important goal as these technologies mature. Here we demonstrate two key steps in this direction. First, a quantum register of Cs single atom qubits is prepared using a 1225-site optical trap array formed with only passive optics, removing the need to use active electro-optic devices for trap pattern formation. The trap pattern is formed using an amplitude mask combined with a Fourier filtering setup, and can be adapted to create dark traps, bright traps, or both interleaved, using only a single trapping wavelength.  Secondly, we show progress towards a rudimentary two-node quantum network of Rb atoms, which is a stepping stone towards the modularization of quantum processors. The network employs nodes with a novel fiber-coupled design and integrated optics, reducing the experimental footprint and leading to superior mechanical stability. We present design and construction techniques used for building the nodes as well as initial results with trapped single atoms.

%not great. probably rewrite

% Quantum technologies are receiving great attention on a global scale, where the quantum computing market is estimated to be worth nearly 1 billion USD\cite{FortuneBusinessInsights2024} at the time of writing this. Despite tremendous progress in the industry toward prototypical devices and early attempts at showing quantum utility, there remain a number of outstanding technical challenges. State of the art quantum computers still take up substantial space, harkening back to the vacuum tube era of classical computing. Reducing the spatial requirements and improving the portability of such machines, by which we mean the ability of these machines to be operated without a team of PhD physicists and engineers on site, is an important goal if this technology is to mature.

% This thesis presents some small steps toward reducing the experimental complexity of neutral atom quantum registers and quantum network nodes. Systems based on atomic qubits, as opposed to the various solid state platforms, are infamous for their room-filling Rube Goldberg machine-like apparatuses, comprised of labrynthine optical paths for a myriad of lasers, often by constructed optic by optic by tired graduate students\footnote{Often just to be rebuilt from scratch the next week, of course}. Although much of the quantum network apparatus described in this thesis can still be described in these terms, the core of the experiment is a fiber-coupled vacuum chamber containing prealigned miniature optics in-vacuo. This drastically reduces the number of components needed for launching six different laser wavelengths at our atoms as well for collecting photons from the atoms with moderate efficiency. This reduces the experimental footprint, enhances mechanical stability, and reduces the overall complexity of the constructed apparatus. This thesis details the work involved to acheive this, the unique engineering hurdles encountered, and initial characterization of the network node with cold atoms. 

% Secondly, we demonstrate a method of trapping atomic qubits in a projected array of optical traps which relies only on passive optical components to generate the trapping pattern. In the same spirit as the fiber-coupled network node, this technique reduces experimental overhead required to trap an atomic register in favor of a less complex solution than the typical methods employed in our field. % needs reworking