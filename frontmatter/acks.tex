% uncomment aks block in frontmatter.tex to show this

% \begin{wbepi}{David C.~Makinson (1965)}
% It is customary for authors of academic books to include in their prefaces statements such as this: ``I am indebted to ... for their invaluable help; however, any errors which remain are my sole responsibility.'' Occasionally an author will go further. Rather than say that if there are any mistakes then he is responsible for them, he will say that there will inevitably be some mistakes and he is responsible for them....

% Although the shouldering of all responsibility is usually a social ritual, the admission that errors exist is not --- it is often a sincere avowal of belief. But this appears to present a living and everyday example of a situation which philosophers have commonly dismissed as absurd; that it is sometimes rational to hold logically incompatible beliefs.
% \end{wbepi}

% Above is the famous ``preface paradox,'' which illustrates how to use the \texttt{wbepi} environment for epigraphs at the beginning of chapters.  You probably also want to thank the Academy.

Spending the majority of days over the course of six and a half years in one room in Chamberlin Hall has been a truly formative experience. The privilege of being able work alongside so many devoted and bright peers has allowed me to learn so much in so short a time, and for that I am truly grateful. Earning this degree has been as much a team effort as an individual one, and every one of my peers here has in some way made a substantive contribution to that.

To my colleagues over the years in the Rubidium lab, or the SNAQ lab, as we have been calling it more recently, I owe you a great deal of gratitude. The experiments that have transpired in that room over my years here have kept us on our toes, often being the thorns in our collective side, but you have stuck with them and stuck with me through all of it. I would like to first thank Minho Kwon, who told me early on that keeping a good spirit was the most challenging part of grad school. Thanks also to Chris Young, whose general knowledge of science, especially entemology, never ceased to impress and entertain during long slogs in the lab. In more recent years, during the quantum network era, I would like to thank Akbar Safari for teaching me so much over the past few years and being a pivotal force in the lab. Your guidance, patience, and encyclopedic knowledge of physics -- in AMO and beyond-- has helped me immensely. To Eunji Oh--