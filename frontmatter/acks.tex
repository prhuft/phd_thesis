% uncomment aks block in frontmatter.tex to show this

% \begin{wbepi}{David C.~Makinson (1965)}
% It is customary for authors of academic books to include in their prefaces statements such as this: ``I am indebted to ... for their invaluable help; however, any errors which remain are my sole responsibility.'' Occasionally an author will go further. Rather than say that if there are any mistakes then he is responsible for them, he will say that there will inevitably be some mistakes and he is responsible for them....

% Although the shouldering of all responsibility is usually a social ritual, the admission that errors exist is not --- it is often a sincere avowal of belief. But this appears to present a living and everyday example of a situation which philosophers have commonly dismissed as absurd; that it is sometimes rational to hold logically incompatible beliefs.
% \end{wbepi}

% Above is the famous ``preface paradox,'' which illustrates how to use the \texttt{wbepi} environment for epigraphs at the beginning of chapters.  You probably also want to thank the Academy.

Spending the majority of days over the course of six and a half years in one room in Chamberlin Hall has been a truly formative experience. The privilege of being able to work alongside so many devoted and bright peers has allowed me to learn so much in so short a time, and for that I am truly grateful. Earning this degree has been as much a team effort as an individual one, and every one of my peers here has in some way made a substantive contribution to that. Additionally, I would like to express deep appreciation for what is beyond Chamberlin Hall, namely the city of Madison. You have so much to offer, from being friendly, endlessly walkable, and filled with parks, to having no shortage of bike paths meandering out into the country. I'm going to miss it. Thanks also to my family for being encouraging and supportive of my endeavors here.

To my colleagues over the years in the Rubidium lab, or the SNAQ lab, as we have been calling it more recently, I owe you a great deal of gratitude. The experiments that have transpired in that room over the years here have kept us on our toes, often being the thorns in our collective side, but you have stuck with them and stuck with me through all of it. I would like to first thank Minho Kwon, who told me early on that keeping a good spirit was the most challenging part of grad school. Thanks also to Chris Young, whose general knowledge of science, especially entemology, never ceased to impress and entertain during long slogs in the lab. In more recent years, during the quantum network era, I would like to thank Akbar Safari for teaching me so much over the past few years and being a pivotal force in the lab. Your guidance, patience, and encyclopedic knowledge of physics -- in AMO and beyond-- has helped me immensely. To Eunji Oh-- thanks for being a great partner in the lab, always making the best of the lab catastrophy du jour. Your diligence, intellect, and ability to problem solve have really helped push things along. I'm sure you'll finish strong. To Gavin Chase, thanks for your general enthusiasm as a team player, and best of luck finishing your circuitous PhD. To Jihwan Moon, I wish you all the best in building up the new Cesium experiment-- make better choices than your predecessors with experiment design. To Omar Nagib-- congrats on escaping work in the lab before it had a chance to significantly slow down your progress towards the PhD. Your theory work has been exciting to follow. I would also like to thank former colleagues Kais Jooya and Yunheung Song for being dedicated partners on the short-lived trap array experiment\footnote{The project was short, and so was the trap lifetime.}.

The QPAL group turned out to be a good place for me to be for a number of reasons, not the least of which was because of the people with whom I have had the pleasure to work. I have appreciated being surrounded by clever, knowledgeable peers who are always willing to lend a hand or help stew on a problem in lab. Thanks in particular to Trent Graham, for sharing tips and tricks from your vast experience, and always being willing to help me think through a physics problem. Thanks also to Ravi Chinnarasu for letting me pick your brain on physics issues and for your help in the early days of the SNAQ project. To Jacob Scott, thanks so much for all of the time you sacrificed helping me track down ground loops in the lab. Thanks to Juan Bohorquez for always being game to escape the lab for dinner and for being my writing partner when we both started our theses. Thanks to Sam for comisserating, shooting the breeze, and providing detailed comments on an early draft of this work. There are too many other people I've crossed paths with here to name them all, but I would like to also thank in no particular order, Jake Uribe, Michael Bergdolt, Z. Alphonse Marra, Sebastian Malewicz, Xiaoyu Jiang, Chris Yip, Jin Zhang, Matt Ebert, and Isaac Scott. Finally, thanks to Mark Saffman, for affording me a generous amount of autonomy and trust in the lab, and for having high standards for my work.

My time here has not passed without being peppered with collaborations and friendships in other lab groups or departments. To Deniz Yavuz and his group, in particular, David Gold, Utku Sa\text{\u{g}}lam, and Matt Beede-- you guys have felt like my cousins in the department, and I've always enjoyed being able to walk down the hall to discuss some issue or exchange equipment. Thanks to Mikhail Kats and his group in ECE, in particular Chengyu Fang and Sanket Deshpande, for thoughtful discussions and for weathering the storm that research often is. Additional thanks to Sanket for all the good times and vibes. Thanks to Randy Goldsmith and his group in Chemistry for putting so much effort into the fiber cavity experiment which I had a short stint on. Thanks to Aidan Tollefson in Medical Physics for being a total copy cat and following me here after undergrad, and for all of the much needed non-physics chats and running advice.

Finally, thanks to Margaret Fortman. You've been there through thick and thin, and its crazy that we've both gotten to go through this together, from start to finish. Your companionship has been indispensible.
