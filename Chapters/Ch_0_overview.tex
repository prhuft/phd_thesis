\part{Theory and Background}\label{part:theory}
\chapter{Overview}\label{ch:overview}

\section{Introduction}
There is abundant interest, in both commercial and academic research settings, in simplifying the vast experimental overhead required for implementing quantum technologies. This is especially for those which fall under the umbrella of atomic, molecular, and optical (AMO) physics, including use of qubits based on neutral atoms and ions, where the optical infrastructure typically takes up an entire room. Besides the spatial consumption, these setups are typically constructed by hand, optic by optic, and are highly reconfigurable. Moving toward more integrated setups will therefore reduce not only space requirements but also improve long-term stability and remove the need to adjust or modify the setup. On the roadmap to more practical quantum computers and networks, this is analogous to going from vacuum tubes to transistors. This thesis will discuss two projects which make steps in this direction, a first-of-its-kind fiber-coupled quantum network based on neutral atoms, and a method for producing scalable optical trap arrays without the need for active electro-optic devices. 

% can repeat the abstract somewhat, but go into more detail
\section{Integrated quantum network nodes}
Experimental AMO platforms are infamous for space consumption and being tedious to construct. They are the Rube Goldberg machines of experimental physics. This thesis describes the construction and characterization of a quantum network node in which the majority of laser beams essential for networking experiments are launched at the atoms using optics which have been prealigned and installed inside the vacuum chamber. The result is that laser cooling, trapping, atomic state preparation, and photon collection are all done with only two of the nine required optical paths being outside of the vacuum chamber, the rest being connected to the outside world by optical fibers fed through the chamber. We will discuss the design, construction, and initial results with cold atoms for the first of two quantum nodes which we have built in this fashion.

\section{Optical trap arrays formed with passive optics}
A second prong of scaling up neutral atom quantum technologies is reducing the complexity of generating optical trap arrays for storing the atomic qubits. Typical approaches involve active devices such as spatial light modulators, consisting of arrays of electronically controlled pixels which modulate either the phase or amplitude of an incident light beam and require computer control to generate trapping patterns. Moreover, these devices typically have limited power handling and limited pixel array sizes, placing an upper bound on the number of traps that can be generated. We discuss here an alternative approach which uses an amplitude mask which can be fabricated on standard optical glass substrates, enabling trap array sizes beyond what can be accomplished today with active devices, while simultaneously reducing experimental complexity. The proposed method is used to demonstrate a 1225-site array of dark traps for Cs atoms, and an extension of the method to trapping two species with only a single wavelength is discussed.